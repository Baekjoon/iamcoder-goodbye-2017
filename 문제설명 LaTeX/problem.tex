% template : olymp.sty (https://github.com/GassaFM/olymp.sty)
% title page from https://github.com/koosaga/iamcoder/blob/master/tests/2016_mockicpc/document.tex

\documentclass[12pt,a4paper,oneside]{article}

\RequirePackage[nonfrench]{kotex}
\usepackage[T2A]{fontenc}
\usepackage[utf8]{inputenc}
\usepackage[english,russian]{babel}
\usepackage{olymp}
\usepackage{graphicx}
\usepackage{amsmath}
\usepackage{amssymb}
\usepackage{array}
\usepackage{color} % for colored text
\usepackage{import} % for changing current dir
\usepackage{epigraph}
\usepackage{daytime} % for displaying version number and date
\usepackage{wrapfig} % for having text alongside pictures
\usepackage{verbatim}

\newcommand{\importproblem}[1]{\import{}{./problems/#1.tex}}
\newcolumntype{M}[1]{>{\centering\arraybackslash}m{#1}}
\newcommand{\blankpage}{
\newpage
\begin{center}
	\hspace{0pt}
	\vfill
	\huge Blank Page
	\hspace{0pt}
	\vfill
\end{center}
}

\contest
{나는코더다 2017 송년대회}%
{2017. 12. 11}%
{19:00 - 23:30}

\binoppenalty=10000
\relpenalty=10000
\exhyphenpenalty=10000

\begin{document}

\begin{titlepage}
	\centering
	{2017년 12월 11일 월요일 19:00 - 23:30\par}
	{경기과학고등학교 33기 김동현, 김현수, 신승원\par}
	\vspace{2cm}
	{\huge 나는코더다 2017 송년 대회 \par}
	\vspace{3cm}
	{\Large 문제 배열은 난이도 순이 아닙니다.\par}
	{\Large 모든 문제를 읽는 것을 권장합니다.\par}
	\vspace{3cm}
	\begin{table}[h]
		\centering
		\renewcommand{\arraystretch}{1.2}
		\begin{tabular}{|l|M{5cm}|M{2cm}|M{2.5cm}|}
			\hline
			& 문제 이름 & 시간 제한 & 메모리 제한 \\ \hline
			A & 경곽 침공 & 5초 & 512MB \\ \hline
			B & 구몬 알고리즘 & 1초 & 512MB \\ \hline
			C & 민돌 투어 & 1초 & 512MB \\ \hline
			D & 비밀 요원 & 2초 & 512MB \\ \hline
			E & 스킬 트리 & 2초 & 512MB \\ \hline
			F & 아티스트 & 1초 & 512MB \\ \hline
			G & 약 팔기 & 1초 & 512MB \\ \hline
			H & 이름 궁합 & 1초 & 512MB \\ \hline
			I & 정과프 해적단 & 1초 & 512MB \\ \hline
			J & 쿵! 쿵! & 2초 & 512MB \\ \hline
			K & 태풍의 아들 KDH & 2초 & 512MB \\ \hline
			L & 현수시티 & 1초 & 512MB \\ \hline
		\end{tabular}
	\end{table}
	
  	\newpage
	\thispagestyle{empty}
	\hspace{0pt}
	\vfill
	\huge Blank Page
	\hspace{0pt}
	\vfill
	
\end{titlepage}

\raggedbottom

\importproblem{A_gshsinvade}

\importproblem{B_kumonalgorithm}

\importproblem{C_mindoltour}

\importproblem{D_secretagent}

\importproblem{E_skilltree}

\importproblem{F_artist}

\importproblem{G_drugseller}

\importproblem{H_namechemi}

\importproblem{I_pirate}

\importproblem{J_kungkung}

\importproblem{K_typhoon}

\importproblem{L_hyunsoocity}

\end{document}

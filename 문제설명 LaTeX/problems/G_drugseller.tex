\begin{problem}{약 팔기}
{}{}
{1초}{512MB}{}

약장수 강욱이는 오늘도 약을 판다. 짬에서 나오는 Vibe로 화려한 언변을 구사하는 강욱이는 최고의 약장수이다. 하지만 이런 그에게도 고민거리가 하나 있으니...

동규라는 단골 손님이 있는데, 그는 매일 약을 $1$알에서 $100$만알 사이의 랜덤한 자연수 갯수만큼 원했다. 주문을 받은 강욱이는 약 상자에서 한 알씩 약을 세서 꺼내주곤 했는데, 그것이 답답했던 동규는 강욱이에게 매번 화를 냈던 것이다.

이러다 동규가 자기를 때리지 않을까 무서웠던 강욱이는 동규가 원하는 만큼의 약을 빨리 건네주기 위한 방법을 고민하기 시작했다. 그는 곧 소싯적에 공부했던 \textbf{Algorithm}을 이용해 다음과 같은 방법을 생각해 냈다.

\textit{`약 봉지 여러 개에 각각 적절한 수의 알약을 담아서 일렬로 늘어 놓은 뒤, 동규가 약을 $k$알 달라고 하면 총 $k$알의 약이 들어있는 어떤 연속한 구간의 약 봉지들을 한 번에 집어 주면 되겠군!'}

아쉽게도, 강욱이의 약 판매대는 봉지를 일렬로 최대 2000개까지만 올려놓을 수 있다. 강욱이는 적은 수의 봉지에 알약을 적절히 담아서 동규가 $100$만 이하의 어떤 수를 부르든 그 수에 해당하는 만큼의 약을 줄 수 있었으면 한다. 하지만 물리 공부를 하느라 \textbf{Algorithm} 공부를 열심히 하지 못한 강욱이는 어떻게 할지 몰라 쩔쩔매고 있다. 강욱이를 도와주자! 

\InputFile

첫 번째 줄에 동규의 최대 약 요구량을 나타내는 정수 N ($=1\, 000\, 000$) 이 주어진다.

\OutputFile

첫 번째 줄에는 필요한 약봉지의 갯수 K ($1 \le K \le 2\, 000$) 를 출력한다.

두 번째 줄에는 왼쪽부터 순서대로 각 약봉지에 들어있어야 하는 약의 수를 출력한다.
각 봉지에는 $1$알 이상 $100$만알 이하의 약이 있어야 한다.

\Examples
	
\begin{example}
\exmp{
6
}{%
3
1 3 2 
}%
\end{example}

\Note

예제 입출력은 실제 문제와 다름에 유의하라.

\blankpage

\end{problem}
